\documentclass{beamer}
\usepackage[utf8]{inputenc}
\usepackage[ngerman]{babel}
\usepackage[T1]{fontenc}
\usepackage{graphicx}
\usepackage[absolute,overlay]{textpos}
\usepackage{csquotes}
\MakeAutoQuote{„}{”}
\usepackage{lmodern}
\usepackage{hyperref}

\hypersetup{
  colorlinks=true,
  linkcolor=[rgb]{1, 1, 1}, %
  urlcolor=[rgb]{.2, .2, .5} %
  % allcolors=[rgb]{0, 0, 0} % schwarz
}

\usetheme{Dresden}
\setbeamertemplate{navigation symbols}{}

\title{Hochschulgruppe für Freie~Software~und~Freies~Wissen}
\subtitle{Datenspuren 2017}
\author{https://fsfw-dresden.de}

\addtobeamertemplate{frametitle}{}{%
  \begin{textblock*}{130mm}(.963\textwidth,8.4mm)
    \includegraphics[width=2.25cm]{img-src/fsfw-logo.pdf}
  \end{textblock*}
}

\begin{document}

\begin{frame}
  \begin{center}%
    \includegraphics[width=4cm]{img-src/fsfw-logo-with-text.pdf}\\

    \vspace*{-0.5\baselineskip}

    \parbox{6cm}{\centering\inserttitle}

    \vspace*{\baselineskip}

    \structure{\Large \insertsubtitle}
  \end{center}
\end{frame}

\begin{frame}[label=ct1]
  \frametitle{Wer sind wir?}

  \onslide<+->

  \begin{itemize}
  \item Hochschulgruppe an der TU (gegründet 2014, ca. 10 P.)
  \item Studierende (TU, HTW) und andere Leute
  \item Hochschulen als Zielgruppe (Multiplikationswirkung)\\
    und Arbeitsfeld (Räume, Strukturen)

    \bigskip\onslide<+->

  \item Bisherige Projekte
    \begin{itemize}
    \item Linux-Install-Party, Linux-Presentation-Day
    \item Verschlüsselungsgewinnspiel
    \item Monatliche Sprechstunde zu \LaTeX{} u.a.
    \item Formulierung eines Programmpapiers
    \item „Uni-Stick”:~80 $\times$ 8\,GB mit freier Software {\tiny zweite Runde diese Woche!}
    \item Git-Workshop bei der OUTPUT
    \end{itemize}
  \end{itemize}
\end{frame}

\begin{frame}[label=ct1b,t]
  \frametitle{Uni-Stick}

  \begin{itemize}
  \item 4000 Flyer in Ersti-Tüten: Gutschein für 8~GB Stick mit freier Software
    fürs Studium, Geld vom TU-StuRa für 80~Stück
  \item Live-Linux / freie Windows-Programme
    \pause
  \item Hat viel Arbeit gemacht
    \pause
  \item Ist gut angekommen (ca. 250 TN)
  \end{itemize}

  \begin{textblock*}{40mm}[0.,0.](80mm,37mm)
    \visible<2->{
      \includegraphics[width=40mm]{img-src/usb-hub}
    }
  \end{textblock*}

  \begin{textblock*}{55mm}[0.,0.](15mm,50mm)
    \visible<3->{
      \includegraphics[width=55mm]{img-src/uni-stick-ausgabe-vortrag}
    }
  \end{textblock*}

\end{frame}

\begin{frame}[label=ct2]
  \frametitle{Warum machen wir das? Aus Überzeugung!}

  \onslide<+->

  \begin{itemize}
  \item Freie und quelloffene Software ist eine nachhaltige Investition in einer digitalen Gesellschaft
    %~ (technische + nicht technische Argumente)\\
    \bigskip\onslide<+->
  \item Freie Lizenzen schützen vor Abhängigkeit
    \bigskip\onslide<+->
  \item Offener Quellcode erhöht die Nachvollziehbarkeit und fördert die technische Bildung
    \bigskip\onslide<+->
  \item Öffentlich finanzierte wissenschaftliche Inhalte sollten nicht
    von Bibliotheken für horrende Summen von Zeitschriften-Verlagen gekauft werden
    müssen
  \end{itemize}
\end{frame}

\begin{frame}[label=ct3]
  \frametitle{Zukunftsideen}

  \begin{itemize}
  \item Studierende zum Nutzen/Verbessern freier Software animieren
    \begin{itemize}
    \item Blog-Beiträge
    \item Kurse (\LaTeX / Python / Git / Inkscape / \dots)
    \item Ansible-Abend
    \item OpenSource-Wettbewerb/Preis
    \end{itemize}

    \bigskip

  \item Aufmerksamkeit erzeugen / Lobby-Arbeit

    \bigskip

  \item Vernetzung mit anderen Städten

  \end{itemize}

\end{frame}

\begin{frame}[label=ct4]
  \frametitle{Public Money, Public Code}

  \begin{textblock*}{40mm}[0.,0.](85mm,25mm)
    \visible<1->{
      \includegraphics[width=40mm]{img-src/PMPC_sticker_v2_55x75}
    }
  \end{textblock*}

  \begin{itemize}
    \item Aktuelle Kampagne der FSF Europe
    \item \textit{Warum wird durch Steuergelder \\finanzierte Software nicht als \\Freie Software veröffentlicht?}
    \item Website: \url{https://publiccode.eu/de/}%
  \end{itemize}

\end{frame}

\begin{frame}[label=ct5]
  \frametitle{Weitere Informationen}

  \onslide<+->

  \begin{center}
    \url{https://fsfw-dresden.de/}
    % HACK THE PLANET!
    $\;\;\left\{\;\;\text{
        \parbox{2.3cm}{
          \texttt{sprechstunde}\\[1mm]
          \texttt{mitmachen}\\[1mm]
          \texttt{fork}\\[1mm]
          \texttt{newsletter}
        }}
    \right.$

    \vspace*{2\bigskipamount}

    \includegraphics[width=50mm]{img-src/fsfw-netzwerke}
  \end{center}

\end{frame}

\end{document}
